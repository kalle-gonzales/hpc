\documentclass[12pt]{report}
\usepackage{xcolor}
\usepackage[german]{babel}
\usepackage[utf8]{inputenc}
\usepackage{fancyhdr}
\usepackage{amsmath}
\usepackage{amssymb}


\parindent=0pt
\parskip=\medskipamount
\pagestyle{fancy}

\begin{document}
\fancyhead[R]{Luca Knaack \& Kilian Mauer}
\fancyhead[L]{HPC - Herleitung Serie 03}
Im Folgenden zeigen wir, wie wir die Summe auf geschickte Weise aufgeteilt berechnen können.
Anstatt für gerade $k$ den Summanden zu addieren und für ungerade $k$ einen recht ähnlichen Summanden zu subtrahieren, 
bilden wir die zunächst die Differenz zwischen den Summanden für gerades $k$ und den darauffolgenden Summanden für ungerades $k$ wie folgt:
\begin{align*}
  &\frac{x^{2k+1}}{(2k+1)!} - \frac{x^{2(k+1)+1}}{(2(k+1)+1)!} \overset{*}{=}\\[5mm]
  &\frac{x^{2k+1}}{(2k+1)!} - \frac{x^{2k+1}*x^2}{(2k+1)! * (2k+2) * (2k+3)} = \\[5mm]
  &\frac{x^{2k+1} * (2k+2) * (2k+3) - x^(2k+1)*x^2}{(2k+1)! * (2k+2) * (2k+3)} =\\[5mm]
  &\frac{x^{2k+1} ((2k+2) * (2k+3) - x^2)}{(2k+1)! * (2k+2) * (2k+3)} =\\[5mm]
  &\frac{(4k^2+ 10k + 6 - x^2)*x^{2k+1}}{(2k+2) * (2k+3) * (2k+1)!} =\\[5mm]
  &\frac{4k^2+ 10k + 6 - x^2}{(2k+2) * (2k+3)} * \frac{x^{2k+1}}{(2k+1)!}
\end{align*}

(*) gilt, da\\
$\frac{x^{2(k+1)+1}}{(2(k+1)+1)!} = \frac{x^{2k+2+1}}{(2k+2+1)!} = \frac{x^{2k+1}*x^2}{(2k+1)! * (2k+2) * (2k+3)}$\\
\end{document}